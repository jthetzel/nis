\nonstopmode{}
\documentclass[a4paper]{book}
\usepackage[times,inconsolata,hyper]{Rd}
\usepackage{makeidx}
\usepackage[utf8,latin1]{inputenc}
% \usepackage{graphicx} % @USE GRAPHICX@
\makeindex{}
\begin{document}
\chapter*{}
\begin{center}
{\textbf{\huge Package `nis'}}
\par\bigskip{\large \today}
\end{center}
\begin{description}
\raggedright{}
\item[Version]\AsIs{0.1-0}
\item[Date]\AsIs{2011-10-18}
\item[Title]\AsIs{Nationwide Inpatient Sample tools for R}
\item[Authors@R]\AsIs{person(given = c(``Jeremy'', ``Thoms''), family = ``Thoms'', email
= ``jthetzel@gmail.com'')}
\item[Author]\AsIs{Jeremy T. Hetzel }\email{jthetzel@gmail.com}\AsIs{}
\item[Maintainer]\AsIs{Jeremy T. Hetzel }\email{jthetzel@gmail.com}\AsIs{}
\item[Depends]\AsIs{R (>= 2.12.0)}
\item[Description]\AsIs{The nis package provides tools for working
with data from the Nationwide Inpatient Sample.
Included is a convenience function to generate SQL code
for importing the NIS flat files into a MySQL database.}
\item[License]\AsIs{GPL-3}
\item[URL]\AsIs{}\url{http://code.google.com/p/nis}\AsIs{}
\item[BugReports]\AsIs{http://code.google.com/p/nis}
\item[LazyLoad]\AsIs{yes}
\item[LazyData]\AsIs{yes}
\item[Collate]\AsIs{'generateSQL.R'}
\end{description}
\Rdcontents{\R{} topics documented:}
\inputencoding{utf8}
\HeaderA{nis-package}{Tools for working with data from the Nationwide Inpatient Sample}{nis.Rdash.package}
%
\begin{Description}\relax
Tools for working with data from the Nationwide Inpatient
Sample
\end{Description}
\inputencoding{utf8}
\HeaderA{generateSQL}{Generate SQL code to convert NIS ascii flat files to SQL database}{generateSQL}
%
\begin{Usage}
\begin{verbatim}
  generateSQL(years, files, type, remove.capitalization =
  T, db.table = NULL, layouts.uri = NULL, old = NULL, new =
  NULL)
\end{verbatim}
\end{Usage}
%
\begin{Arguments}
\begin{ldescription}
\item[\code{years}] A vector containing the years of data to
include. The order of \code{years} must be the same as
file locations provided in the \code{files} parameter.
Currently, only years 1998 through 2009 are supported.

\item[\code{files}] A vector containing the file locations of
the ascii flat files.  The order of \code{files} must be
the same as the years provided by the \code{years}
parameter.

\item[\code{type}] The type of NIS data being loaded.
Acceptable values are \code{"core"}, \code{"hospitals"},
\code{"severity"}, and \code{"groups"}, for the NIS Core,
Hospitals, Severity, and Dx Pr Groups data files,
respectively.

\item[\code{remove.capitalization}] optional A logical
indicating whether the variable names should be converted
to all lower case. Default is \code{TRUE}.

\item[\code{db.table}] optional A character indicating the
desired database table name. If not specified, will
default to the \code{type} parameter (i.e. core,
hospitals, severity, or groups)

\item[\code{layouts.uri}] optional A list specifying the uris to
use to fetch the layouts of the ascii flat files. This
should usually not be specified, in which case the
default uris are used, as listed in details below.

\item[\code{old}] optional A vector of variable names to be
replaced by \code{new}. Must be in the same order as the
\code{new} parameter.

\item[\code{new}] optional A vector of new variable names to
replace those specified in \code{old}. Must be in the
same order as the \code{old} parameter.
\end{ldescription}
\end{Arguments}
%
\begin{Value}
A list of two character vectors: \begin{ldescription}
\item[\code{createTable}] A SQL
statement to create an empty table with with variables
appropriate for NIS data.\item[\code{loadData}] One or more
SQL statements to load data from the specified
\code{files} into a MySQL table using the \code{load data
  infile} SQL statement.

\end{ldescription}

The \code{generateSQL} function is a convenience function
that returns SQL code to be used for creating appropriate
tables for a MySQL database and loading data from the
flat ascii files provided by HCUP.

Details
Unless \code{layouts.uri} is specified,
\code{generateSQL} uses the parses layout information
from the NIS website to determine the appropriate
variable names and fixed-width locations of data in the
NIS ascii flat files. By default, the \code{layouts.uri}
object is the following list: \code{layouts.uri <- list(
  "1998" =
  'http://www.hcup-us.ahrq.gov/db/nation/nis/tools/stats/NIS\_1998\_COREv2.TXT',
  "1999" =
  'http://www.hcup-us.ahrq.gov/db/nation/nis/tools/stats/NIS\_1999\_COREv2.TXT',
  "2000" =
  'http://www.hcup-us.ahrq.gov/db/nation/nis/tools/stats/NIS\_2000\_CORE.TXT',
  "2001" =
  'http://www.hcup-us.ahrq.gov/db/nation/nis/tools/stats/NIS\_2001\_CORE.TXT',
  "2002" =
  'http://www.hcup-us.ahrq.gov/db/nation/nis/tools/stats/FileSpecifications\_NIS\_2002\_CORE.TXT',
  "2003" =
  'http://www.hcup-us.ahrq.gov/db/nation/nis/tools/stats/FileSpecifications\_NIS\_2003\_CORE.TXT',
  "2004" =
  'http://www.hcup-us.ahrq.gov/db/nation/nis/tools/stats/FileSpecifications\_NIS\_2004\_CORE.TXT',
  "2005" =
  'http://www.hcup-us.ahrq.gov/db/nation/nis/tools/stats/FileSpecifications\_NIS\_2005\_Core.TXT',
  "2006" =
  'http://www.hcup-us.ahrq.gov/db/nation/nis/tools/stats/FileSpecifications\_NIS\_2006\_Core.TXT',
  "2007" =
  'http://www.hcup-us.ahrq.gov/db/nation/nis/tools/stats/FileSpecifications\_NIS\_2007\_Core.TXT',
  "2008" =
  'http://www.hcup-us.ahrq.gov/db/nation/nis/tools/stats/FileSpecifications\_NIS\_2008\_Core.TXT',
  "2009" =
  'http://www.hcup-us.ahrq.gov/db/nation/nis/tools/stats/FileSpecifications\_NIS\_2009\_Core.TXT'
  )}


## Not run: 
\#\# Create SQL statement to upload NIS Core data from 1998-2009 to MySQL database
\# Years to include
years <- c(seq(1998, 2009))

\# Location of the NIS ascii flat files
files <- c('/data/nis/NIS\_1998\_Core.ASC',
	'/data/nis/NIS\_1999\_Core.ASC',
	'/data/nis/NIS\_2000\_Core.ASC',
	'/data/nis/NIS\_2001\_Core.ASC',
	'/data/nis/NIS\_2002\_Core.ASC',
	'/data/nis/NIS\_2003\_Core.ASC',
	'/data/nis/NIS\_2004\_Core.ASC',
	'/data/nis/NIS\_2005\_Core.ASC',
	'/data/nis/NIS\_2006\_Core.ASC',
	'/data/nis/NIS\_2007\_Core.ASC',
	'/data/nis/NIS\_2008\_Core.ASC',
	'/data/nis/NIS\_2009\_Core.ASC'
)

\# Generate SQL statemebts
sql.core <- generateSQL(years = years, files = files, type = "core")

\# Output SQL statements to a file
cat(sql.core\$createTable, file = "makeTableCore.sql")
cat(sql.core\$uploadData, file = "uploadDataCore.sql")

## End(Not run)


http://www.hcup-us.ahrq.gov/nisoverview.jsp


\end{Value}
\printindex{}
\end{document}
